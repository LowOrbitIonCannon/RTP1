\documentclass{article}

\usepackage[ngerman]{babel}										% Statt german, \autoref zeigt damit "Abbildung x" an
\usepackage{graphicx}											% Graphiken einbinden
\usepackage[left=3cm,right=3cm,top=3cm,bottom=3cm]{geometry}	% Abstände zu den Papierrändern
\usepackage[utf8]{inputenc}										% Zeichenkodierung
\usepackage{mathtools}											% Stellt Mathe-Umgebung bereit
\usepackage{float}												% Improves floating elements
\usepackage{lmodern}											% Andere Schriftart (bessere Druckbarkeit)
\usepackage{pdfpages}											% Einbinden von PDF-Dokumenten
\usepackage{fancyhdr}											% Fancy Headers / Footers
\usepackage{listliketab}										% Lists with tab stops
\usepackage{setspace}											% Anpassung von Zeilenabständen
\usepackage{siunitx}											% Korrekte Darstellungen von Einheiten (\SI{1.5}{\milli\volt})
\usepackage{hyperref}   										% Objekte referenzieren (\autoref)
\usepackage{subcaption}
\usepackage{listings}
\DeclareSIUnit{\Bit}{Bit} % bits 

% dark mode
%\usepackage{xcolor}			
%\pagecolor[rgb]{0,0,0}
%\color[rgb]{1,1,1}

\setlength{\parindent}{0em}										% Keine Einrückung bei Absatzbeginn
\setlength{\parskip}{1em}										% Dafür vert. Abstand zw. Absätzen

\renewcommand{\arraystretch}{1.2}								% Verzeichnisse kompakter darstellen

\newcommand{\tbf}{\textbf}										% Shortcut für fetten Text

\makeatletter
\g@addto@macro\bfseries{\boldmath}								% \mathbf nicht im Inhaltsverzeichnis darstellen
\makeatother

\pagestyle{fancy}												% Benutzung von fancyhdr
\sisetup{
	locale = DE,
	per-mode=fraction,
	fraction-function=\tfrac
	}
