\section{Einleitung}

Im diesem Versuch sollen die Eigenschaften einer Regelstrecke untersucht werden. Die Regelstrecke besteht aus einem Lüfter, einer Heizung, einem Temperatur- und einem Drucksensor. Für diesen Versuch sind zunächst die Auswirkungen von Lüfterdrehzahl- und Heizleistungsänderungen auf die Lufttemperatur am Ausgang interessant. Es werden sowohl Messungen im stationären Zustand als auch mit dynamischen Eingangsignalen gemacht. Ziel der Messungen ist es, ein akkurates mathematisches Modell der Regelstrecke zu erstellen. Dafür werden Sprungantworten aufgezeichnet und anschließend numerisch an ein entsprechendes Modell gefittet.

%\subsection{Verwendete Geräte/Software}

%Für den Versuch wird folgende Geräte/Software verwendet:

%\begin{table}[ht]
%    \centering
%    \begin{tabular}{|c|c|c|c|}\hline
%    \tbf{Gerätetyp}     & \tbf{Bezeichnung} \\ \hline
%                        &                   \\ \hline
%                        &                   \\ \hline
%    \end{tabular}
%    \caption{Auflistung der Geräte/Software}
%\end{table}
